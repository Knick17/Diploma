\documentclass[a4paper,12pt]{article}
\usepackage[utf8]{inputenc}
\usepackage[russian]{babel}
\usepackage{tocloft}
\usepackage{setspace}
\usepackage{geometry}
\geometry{margin=2.5cm}
\setlength{\parindent}{0pt}
\renewcommand{\cftsecleader}{\cftdotfill{\cftdotsep}}

\begin{document}

\tableofcontents

\section*{ВВЕДЕНИЕ}

\section{ТЕОРЕТИЧЕСКИЕ ОСНОВЫ СТАТИСТИЧЕСКОГО АНАЛИЗА И МАШИННОГО ОБУЧЕНИЯ}

\subsection{Статистические критерии и методы проверки гипотез}
\subsubsection{Корреляция Кендалла-тау}
\subsubsection{Корреляция Пирсона}
\subsubsection{Корреляция Спирмена}

\subsection{Доверительные интервалы для доли (бернуллиевой величины)}
\subsubsection{Классические подходы построения интервалов}
\paragraph{Wilson score interval}
\paragraph{Jeffreys interval}
\paragraph{Clopper–Pearson interval}
\paragraph{Agresti–Coull interval}
\subsubsection{Сравнение и обсуждение доверительных интервалов}
\subsubsection{Примеры вычисления доверительных интервалов}

\subsection{Примеры вычисления коэффициентов корреляции}
\subsection{Примеры применения статистических критериев для проверки гипотез о значимости коэффициентов корреляции}

\subsection{Методы и модели машинного обучения}
\subsubsection{Логистическая регрессия: теоретические аспекты и интерпретация результатов}
\subsubsection{Модель градиентного бустинга CatBoost}
\subsubsection{Интерпретация моделей машинного обучения с помощью SHAP}

\section{ПРАКТИЧЕСКИЙ АНАЛИЗ ВЛИЯНИЯ ОПЕРАЦИОННЫХ МЕТРИК НА УРОВЕНЬ CSAT}

\subsection{Подготовка и описание данных}
\subsubsection{Описание используемой витрины данных}
\subsubsection{Операционные и неоперационные признаки в исследовании}

\subsection{Исследование данных с использованием статистических критериев}
\subsubsection{Анализ зависимости CSAT от времени жизни таска}
\subsubsection{Анализ влияния различных факторов на оценки пользователей}

\subsection{Построение и анализ моделей машинного обучения}
\subsubsection{Построение логистической регрессии и анализ ее коэффициентов}
\subsubsection{Применение модели CatBoost и анализ важности признаков с помощью SHAP}

\section{РЕЗУЛЬТАТЫ И ВЫВОДЫ ИССЛЕДОВАНИЯ}
\subsection{Оценка эффективности сокращения времени выполнения задач}
\subsection{Интерпретация и практические рекомендации на основе моделей}

\section*{ЗАКЛЮЧЕНИЕ}
\section*{СПИСОК ИСТОЧНИКОВ}

\end{document}
