\renewcommand{\section}[2]{\anonsection{Библиографический список}}
%%% Подключение списка литературы по госту 2008 года
\bibliographystyle{ugost2008l}
% Подключаем сам файл с литературой (разрешение .bib)

%%% В ФАЙЛЕ ДЛЯ РУССКИХ ИСТОЧНИКОВ ДОБАВЛЯЕМ language={russian}, 
%%% ДЛЯ УКРАИНСКИХ language={ukrainian}. ТОГДА БУДЕТ ПРЯМ ХОРОШО.
%\bibliography{references}

\begin{thebibliography}{9}

\bibitem{kenda1938}
Kendall, M. G. (1938). A new measure of rank correlation. \emph{Biometrika}, 30(1/2), 81–93. \url{https://doi.org/10.2307/2332226}

\bibitem{pearson1896}
Pearson, K. (1896). Mathematical contributions to the theory of evolution. III. Regression, heredity, and panmixia. \emph{Philosophical Transactions of the Royal Society of London A}, 187, 253–318. \url{https://doi.org/10.1098/rsta.1896.0007}

\bibitem{spearman1904}
Spearman, C. (1904). The proof and measurement of association between two things. \emph{The American Journal of Psychology}, 15(1), 72–101. \url{https://doi.org/10.2307/1412159}

\bibitem{bonett2000}
Bonett, D. G., \& Wright, T. A. (2000). Sample size requirements for estimating Pearson, Kendall and Spearman correlations. \emph{Psychometrika}, 65(1), 23–28. \url{https://doi.org/10.1007/BF02294183}

\bibitem{fisher1921}
Fisher, R. A. (1921). On the probable error of a coefficient of correlation deduced from a small sample. \emph{Metron}, 1, 3–32. Reprinted in \emph{Collected Papers of R. A. Fisher}.

\bibitem{zimmerman1997}
Zimmerman, D. W., \& Zumbo, B. D. (1997). Relative power of the Wilcoxon test, the Friedman test, and repeated-measures ANOVA on ranks. \emph{Journal of Experimental Education}, 65(1), 71–86. \url{https://doi.org/10.1080/00220973.1996.9943798}


\bibitem{brown2001interval}
Brown, L. D., Cai, T. T., \& DasGupta, A. (2001). Interval estimation for a binomial proportion. \textit{Statistical Science}, 16(2), 101–133. \url{https://doi.org/10.1214/ss/1009213286}

\bibitem{clopper1934use}
Clopper, C. J., \& Pearson, E. S. (1934). The use of confidence or fiducial limits illustrated in the case of the binomial. \textit{Biometrika}, 26(4), 404–413. \url{https://doi.org/10.1093/biomet/26.4.404}

\bibitem{agresti1998}
Agresti, A., \& Coull, B. A. (1998). Approximate is better than “exact” for interval estimation of binomial proportions. \textit{The American Statistician}, 52(2), 119–126. \url{https://doi.org/10.1080/00031305.1998.10480550}

\end{thebibliography}


%\label{sec:bibliography}

% Обязательно добавляем это в конце каждой секции, чтобы 
% обеспечить переход на новую страницу
\clearpage