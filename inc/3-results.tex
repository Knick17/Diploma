%%% Результаты %%%
\section{ПРАКТИЧЕСКИЙ АНАЛИЗ ВЛИЯНИЯ ОПЕРАЦИОННЫХ МЕТРИК НА УРОВЕНЬ CSAT}

\subsection{Подготовка и описание данных}
\subsubsection{Описание используемой витрины данных}
\subsubsection{Операционные и неоперационные признаки в исследовании}

\subsection{Исследование данных с использованием статистических критериев}
\subsubsection{Анализ зависимости CSAT от времени жизни таска}
\subsubsection{Оценка необходимости проведения A/B теста}

\subsection{Построение и анализ моделей машинного обучения}
\subsubsection{Построение логистической регрессии и анализ ее коэффициентов}
\subsubsection{Применение модели CatBoost и анализ важности признаков с помощью SHAP}
\subsubsection{Анализ влияния различных факторов на оценки пользователей}


% Обязательно добавляем это в конце каждой секции, чтобы 
% обеспечить переход на новую страницу
\clearpage