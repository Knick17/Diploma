\documentclass[12pt]{article}
\usepackage[utf8]{inputenc}
\usepackage[russian]{babel}
\usepackage{amsmath}
\usepackage{graphicx}
\usepackage{hyperref}
\usepackage{titlesec}
\usepackage{geometry}
\geometry{a4paper, margin=2.5cm}
\titleformat{\section}{\normalfont\Large\bfseries}{\thesection}{1em}{}
\titleformat{\subsection}{\normalfont\large\bfseries}{\thesubsection}{1em}{}
\titleformat{\subsubsection}{\normalfont\normalsize\bfseries}{\thesubsubsection}{1em}{}

\title{\textbf{Исследование влияния операционных метрик на метрики удовлетворенности клиента}}
\author{}
\date{}

\begin{document}

\maketitle
\thispagestyle{empty}
\newpage

\tableofcontents
\thispagestyle{empty}
\newpage

\section{Введение}

\subsection*{Цель исследования}
\vspace{0.5em}

\subsection*{Задачи исследования}
\vspace{0.5em}

\subsection*{Актуальность темы}
\vspace{0.5em}

\subsection*{Объект и предмет исследования}
\vspace{0.5em}

\subsection*{Методы исследования}
\vspace{0.5em}

\newpage
\section{Обзор литературы и статистические критерии анализа связи между переменными}

\subsection{Анализ современной литературы по изучаемой теме}

\subsubsection{Подходы к измерению удовлетворенности клиентов}
\vspace{0.5em}

\subsubsection{Операционные метрики в управлении качеством обслуживания}
\vspace{0.5em}

\subsection{Статистические критерии для проверки гипотез о зависимости метрик}

\subsubsection{Binomial test}
\paragraph{Теоретическое описание}
\vspace{0.5em}

\paragraph{Пример использования}
\vspace{0.5em}

\subsubsection{Критерий Cramér's V}

\paragraph{Теоретическое описание}
\vspace{0.5em}

\paragraph{Пример использования}
\vspace{0.5em}

\subsubsection{Критерий Пирсона ($\chi^2$)}

\paragraph{Теоретическое описание}
\vspace{0.5em}

\paragraph{Пример использования}
\vspace{0.5em}

\subsubsection{Критерий корреляции Спирмена}

\paragraph{Теоретическое описание}
\vspace{0.5em}

\paragraph{Пример использования}
\vspace{0.5em}

\subsubsection{Критерий корреляции Пирсона}

\paragraph{Теоретическое описание}
\vspace{0.5em}

\paragraph{Пример использования}
\vspace{0.5em}

\subsubsection{ANOVA}
\vspace{0.5em}

\paragraph{Пример использования}
\vspace{0.5em}

\newpage
\section{Методология}

\subsection{Выбор методов исследования}
\vspace{0.5em}

\subsection{Описание этапов исследования}
\vspace{0.5em}

\subsection{Обоснование подхода}
\vspace{0.5em}

\newpage
\section{Эмпирическая часть}

\subsection{Представление данных: описание источника, объема и структуры информации}
\vspace{0.5em}

\subsection{Проведение анализа: выявление корреляций, построение регрессионных моделей, визуализация взаимосвязей}
\vspace{0.5em}

\subsection{Интерпретация результатов: какие операционные метрики оказывают наибольшее влияние на удовлетворенность клиентов}
\vspace{0.5em}

\subsection{Примеры кейсов}
\vspace{0.5em}

\newpage
\section{Заключение}

\subsection{Сводные выводы по результатам исследования}
\vspace{0.5em}

\subsection{Обоснование практической значимости}
\vspace{0.5em}

\subsection{Рекомендации по управлению ключевыми метриками}
\vspace{0.5em}

\subsection{Предложения для дальнейших исследований}
\vspace{0.5em}

\newpage
\section{Список использованной литературы}

\subsection{Научные статьи, книги, аналитические отчеты, нормативные документы, статистические источники, примененные в работе}
\vspace{0.5em}

\newpage
\section{Приложения}

\begin{itemize}
    \item Таблицы с исходными данными
    \item Результаты анализа (графики, диаграммы, модели)
    \item Формулы и расчеты, используемые в эмпирической части
    \item Примеры анкет или методик измерения метрик (если использовались)
\end{itemize}

\end{document}
